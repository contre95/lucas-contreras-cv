%-------------------------------------------------------------------------------
%	SECTION TITLE
%-------------------------------------------------------------------------------
\cvsection{Experiencia laboral}

%-------------------------------------------------------------------------------
%	CONTENT
%-------------------------------------------------------------------------------
\begin{cventries}
	%---------------------------------------------------------
	% \cventry{<position>}{<title>}{<location>}{<date>}{<description>}
	\cventry
	{Ingeniero cyberseguridad cloud IV} % Job title
	{N26} % Organization
	{Barcelona, Spain} % Location
	{Apr. 2023 - Presente} % Date(s)
	{
		\begin{cvitems} % Description(s) of tasks/responsibilities
			\item {Actualmente estoy trabajando en el proyecto de gestión de vulnerabilidades, responsable del diseño de arquitecture y colabolarndo con la implementación.}
		\end{cvitems}
	}

	\cventry
	{Ingeniero cyberseguridad cloud III} % Job title
	{} % Organization
	{} % Location
	{Sep. 2021 - Abr. 2023} % Date(s)
	{
		\begin{cvitems} % Description(s) of tasks/responsibilities
			\item {Liderar la migración del sistema SIEM de N26 basado en el stack ELK a servicios administrados de AWS como ECS y Opensearch, lo que resultó en una reducción significativa del mantenimiento. Al mismo tiempo, se establecieron métricas de aplicaciones como SLA, SLI y SLO, y se implementaron sistemas de monitoreo complementarios junto con estrategias de autocorrección y automatizaciones. Lo que redujo la fricción con nuestras partes interesadas y mejoró drásticamente nuestro MTTR.}
			\item {Desarrollé Philips, un sistema que realiza un seguimiento de la propiedad de la propiedad intelectual en todos los equipos, servicios y recursos de AWS junto con los plazos correspondientes, una herramienta que se utiliza para ayudar en los análisis forenses posteriores. Para esta herramienta utilicé Python, DynamoDB y MSK (Kafka). Durante su desarrollo se tuvieron en cuenta aspectos como pruebas unitarias y un enfoque consciente hacia la extensibilidad del sistema, adoptando elementos de Domain Driven Design y Plugin Architecture.}
		\end{cvitems}
	}

	\cventry
	{SR. Ingeniero cyberseguridad cloud} % Job title
	{Mercadolibre} % Organization
	{Argentina} % Location
	{Jul. 2021 - Sep. 2021} % Date()
	{
		\begin{cvitems} % Description(s) of tasks/responsibilities
			\item {Dentro de un grupo de dos personas construimos POLP Fiction, una herramienta que tiene como objetivo aplicar el principio de privilegio mínimo en las políticas, usuarios y roles administrados por el cliente de AWS IAM. El proyecto logró reducir la superficie de ataque en más de 200 cuentas de AWS identificando y mitigando problemas de seguridad como escalada de privilegios, confused deputy problem y privilegios mínimos. Hicimos el inventario de políticas de código abierto https://github.com/mercadolibre/polp-fiction-metrics}
		\end{cvitems}
	}
	\cventry
	{SSR. Ingeniero Cyberseguridad cloud} % Job title
	{} % Organization
	{} % Location
	{Nov. 2019 - Jul. 2021} % Date()
	{
		\begin{cvitems} % Description(s) of tasks/responsibilities
			\item {En asociación con otro ingeniero, participé en la solución Patch Management de Mercadolibre, un proyecto que utilizando Lambda, AWS Config y Systems Manager maneja el parcheo de +50k EC2 e instancias informáticas en AWS y GCP respectivamente en alrededor de +5 distribuciones de Linux diferentes.}
			\item {Como una de mis tareas más importantes en la iniciativa de administración de parches, creé una herramienta que aprovecha IAP (GCP Identity Aware Proxy) de Google y Ansible para instalar el agente AWS SSM en cualquier instancia de GCP, lo que hace que la solución de parches sea multinube y administrada de manera centralizada.}
		\end{cvitems}
	}

	%---------------------------------------------------------
	\cventry
    {Site Reliability Engineer (SRE)} % Job title
	{Edrans} % Organization
	{Argentina - USA} % Location
	{Jun. 2018 - Nov. 2019} % Date(s)
	{
		{Trabajé como contratista principalmente para los comercios electrónicos Zappos y OLX en los cuales:}
		\linebreak
		\begin{cvitems} % Description(s) of tasks/responsibilities
			\item {Desarrollé una herramienta completa de respuesta a incidentes sin servidor con 3 microservicios/API totalmente componibles y reutilizables utilizando Python y NodeJS en AWS. Herramientas que luego fueron utilizadas por el equipo de respuesta a incidentes de Zappos.}
			\item {Se resolvieron varios problemas de IaC con Terraform y CloudFormation en AWS y se automatizaron algunos procedimientos utilizando Docker, Lambda y más servicios nativos de AWS.}
			\item {Implementé stacks de monitoreo en contenedores como Nagios, TIG (Telegraf, InfluxDB y Grafana)}
		\end{cvitems}
	}

	% %---------------------------------------------------------
	% \cventry
	% {Freelance Developer} % Job title
	% {Codelamp} % Organization
	% {Argentina} % Location
	% {Sep. 2013, Sep. 2021} % Date(s)
	% {
	% \begin{cvitems}
	%     \item {Dealt with most aspects of software development, from writing a functional brief to designing, coding, selling and maintaining a system.}
	% \end{cvitems}
	% }

	%%---------------------------------------------------------
	%\cventry
	%{Full Stack Developer}  Job title
	%{Go Soluciones y Sistemas}  Organization
	%{Argentina - Uruguay}  Location
	%{May. 2018 - Jun. 2018}  Date(s)
	%{
	%\begin{cvitems}  Description(s) of tasks/responsibilities
	%\item {Developed several modules of a system that tracks trucks and sends a report over GPS signal in order to calculate trips' costs depending on different variables and had to interact with the SAP API in order generate ready-to-print invoices from within the system I was developing.}
	%\end{cvitems}
	%}

	%---------------------------------------------------------
	\cventry
	{Security Officer} % Job title
	{Toyota S.A} % Organization
	{Argentina} % Location
	{Dec. 2016 - Dec. 2017} % Date(s)
	{
		\begin{cvitems} % Description(s) of tasks/responsibilities
			\item {Como parte de mi pasantía en Toyota, administré Active Directory, MySQL y Windows Servers, así como procedimientos de instalación masiva de software con el software InvGate.}
			\item {En su mayor parte, aprendí mucho sobre el estilo Toyota, TPS, el círculo Kaizen de Toyota para la mejora continua y varias metodologías fascinantes basadas en datos que Toyota aplica para lograr resultados sobresalientes.}
		\end{cvitems}
	}
	%---------------------------------------------------------
\end{cventries}
