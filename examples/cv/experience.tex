%-------------------------------------------------------------------------------
%	SECTION TITLE
%-------------------------------------------------------------------------------
\cvsection{Work Experience}

%-------------------------------------------------------------------------------
%	CONTENT
%-------------------------------------------------------------------------------
\begin{cventries}
%---------------------------------------------------------
% \cventry{<position>}{<title>}{<location>}{<date>}{<description>}
\cventry
{Cloud Security Engeneering IV} % Job title
{N26} % Organization
{Barcelona, Spain} % Location
{Sep. 2021 - Present} % Date(s)
{
\begin{cvitems} % Description(s) of tasks/responsibilities
\item {Serving as a internal team systemct engineer function tech lead.}
\item {Currently working on the Vulnerability management project, helping with it's design and implementation.}
\end{cvitems}
}

\cventry
{Cloud Security Engeneering III} % Job title
{} % Organization
{} % Location
{Sep. 2021 - Present} % Date(s)
{
\begin{cvitems} % Description(s) of tasks/responsibilities
\item {Drove the migration of our custom SIEM system based on the ELK stack to AWS managed services, resulting in a sifnificant reduction in maintenance. Simultaneously, established application metrics such as SLA, SLI, and SLO, and implemented supplementary monitoring systems alongside self-remediation strategies and automations. Which reduced friction with our stakeholders and drastically improved our MTTR.}
\item {I developed Philips, a system which keeps track of IP ownership across teams, services, and AWS resources along with corresponding timeframes, a tool used to help subsequent forensic analyses. For this tool I used Python, DynamoDB and MSK (Kafka). The following was taken into account during it's development, unit testing, and a mindful approach towards the system's extensibility. The methodology embraced elements from Domain Driven Design and Plugin Architecture.}
\end{cvitems}
}

\cventry
{SR. Cloud Security Engeneering} % Job title
{Mercadolibre} % Organization
{Argentina} % Location
{Nov. 2019 - Sep 2021} % Date()
{
\begin{cvitems} % Description(s) of tasks/responsibilities
\item {Within a group of two people we built POLP Fiction, a tool that aims to apply the principle of least privilege on AWS IAM Customer managed Policies, Users and Roles. The project managed to reduced the attack surface on more than 200 AWS tackling identifying and mitigating security concerns such as privilege escalation and confused deputy problem and least privilege. We made the Policy inventory open source https://github.com/mercadolibre/polp-fiction-metrics}
\end{cvitems} 
}
\cventry
{SSR. Cloud Security Engeneering} % Job title
{} % Organization
{} % Location
{Nov. 2019 - Sep 2021} % Date()
{
\begin{cvitems} % Description(s) of tasks/responsibilities
\item {As part of working group of two engineers, I participated in Mercadolibre's Patch Management solution, A project that using Lambda, AWS Config, and System Manager, that handles the patching of +50k EC2 and compute instances on AWS and GCP respectively across around 5+ different Linux distributions.}
\item {As part of my duties in the patch management initiative, I created an tool that leverages on Google's IAP (GCP Identity Aware Proxy) and Ansible to install the AWS SSM agent on any GCP instance, making our patching solution multi cloud and centrally managed.}
\end{cvitems} 
}

%---------------------------------------------------------
\cventry
{Site Reliability Engineer} % Job title
{Edrans} % Organization
{Argentina - USA} % Location
{Jun. 2018 - Nov. 2019} % Date(s)
{
\begin{cvitems} % Description(s) of tasks/responsibilities
\item {Worked as a contractor mainly for e-commerces Zappos and OLX in which I:}
\item {Developed a full Serverless incident response tool with 3 fully composable and reusable Microservices/APIs using Python and NodeJS in AWS. Tooling which was later on used by Zappos Incident response team.}
\item {Solved several IaC issues with Terraform and CloudFormation in AWS and automate a handful of procedures using Docker, Lambda and more native AWS services}
\item {Deploy conteinarized monitoring stacks such as as Nagios, TIG (Telegraf, InfluxDB, and Grafana)}
\end{cvitems}
}

% %---------------------------------------------------------
% \cventry
% {Freelance Developer} % Job title
% {Codelamp} % Organization
% {Argentina} % Location
% {Sep. 2013, Sep. 2021} % Date(s)
% {
% \begin{cvitems}
%     \item {Dealt with most aspects of software development, from writing a functional brief to designing, coding, selling and maintaining a system.}
% \end{cvitems}
% }

%%---------------------------------------------------------
%\cventry
%{Full Stack Developer}  Job title
%{Go Soluciones y Sistemas}  Organization
%{Argentina - Uruguay}  Location
%{May. 2018 - Jun. 2018}  Date(s)
%{
%\begin{cvitems}  Description(s) of tasks/responsibilities
%\item {Developed several modules of a system that tracks trucks and sends a report over GPS signal in order to calculate trips' costs depending on different variables and had to interact with the SAP API in order generate ready-to-print invoices from within the system I was developing.}
%\end{cvitems}
%}

%---------------------------------------------------------
\cventry
{Security Officer} % Job title
{Toyota S.A} % Organization
{Argentina} % Location
{Dec. 2016 - Dec. 2017} % Date(s)
{
\begin{cvitems} % Description(s) of tasks/responsibilities
\item {As part of my internship at Toyota, I Administrated Active Directory, MySQL and Windows Servers as well as massive software installation procedures with InvGate software.}
\item {For the most part I learned a lot about The Toyota way, TPS, Toyotas Kaizen circulo for continuous improvement and several fascinating data driveen methodologies that Toyota's applies to achieve outstanding results.}
\end{cvitems}
}
%---------------------------------------------------------
\end{cventries}
